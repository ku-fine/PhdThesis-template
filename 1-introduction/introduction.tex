\chapter{Introduction}
\label{cha:introduction}

\textit{Here are a few expert from my thesis to populate the Table of Contents and demonstrate basic reference and citation commands.}

\section{Distributed \& parallel computing}

Modern supercomputers feature an unprecedented level of parallelism.
For instance, the current flagship supercomputer of Japan, Fugaku, is composed of over 158 thousand compute nodes, each node containing a many-core 48 core processor~\cite{fugaku}.
Harnessing the processing power of such large systems poses a challenge to application developers.

Borrowing the definition from van Steen and Tanenbaum~\cite{distributedsystems}:
\begin{quote}
A distributed system is a collection of autonomous computing elements that appears to its users as a single coherent system.
\end{quote}
Programming models for distributed programming should therefore present a single coherent system for programmers to be able to reason and create distributed programs with ease.
For the sake of example, here are two more citations comprising some conference proceedings and journal article~\cite{habanerox10,resilientx10}.

\begin{figure}
	\centering
\includegraphics[trim={3.5cm, 10.8cm, 4.2cm, 3cm},clip,width=.3\textwidth]{1-introduction/nqueens}
\caption{One of 92 solutions to the 8-Queens problem}
\label{fig:nqueenssolution}
\end{figure}

A distributed \emph{Hello World} program demonstrating the use of the \lstinline|finish| and \lstinline|asyncAt| constructs is presented in Listing~\ref{lst:javahello}.
A possible output of this program shown in Listing~\ref{lst:javahellooutput}.
In this example, the \verb|finish| method is called on line~7 to~11, with an asynchronous activity spawned on each place participating in the computation using a for loop and the \lstinline|asyncAt| method on line~9. 
The program will not progress beyond the \lstinline|finish| until every place prints its ``hello'' message.

\begin{lstlisting}[float,caption={Distributed Hello World in Java},label={lst:javahello}]
	import static apgas.Constructs.*;
	import apgas.Place;
	class HelloWorld {
		public static void main(String[] args) {
			System.out.println("Running main at " + here() + " of " + 
			places().size() + " places");
			finish(() -> {
				for (Place p : places()) {
					asyncAt(p, () -> System.out.println("Hello from " + here()));
				}
			});
			System.out.println("Bye");
		}
	}
\end{lstlisting}

\begin{lstlisting}[float,caption={Possible output of the Listing~\ref{lst:javahello} program running with 4 processes},label={lst:javahellooutput}]
	Running main at place(0) of 4 places
	Hello from place(0)
	Hello from place(3)
	Hello from place(2)
	Hello from place(1)
	Bye
\end{lstlisting}


The hardware and software configuration of the cluster and supercomputer used in our evaluations are summarized in Table~\ref{tbl:ofp}.

\begin{table}[ht]
	\centering
	\caption{Characteristics of the OakForest-PACS supercomputer}
	\label{tbl:ofp}
	\begin{tabular}{ll}
		\toprule
		Server Type & Fujitsu PRIMERGY CX1640 M1 \\
		\midrule
		Nb of nodes & 8208 \\
		Processor & Intel Xeon Phi 7250 (1.4 GHz, 68 cores) \\
		RAM & 96GB DDR4 \\
		Interconnection & Intel Omni-Path (100 Gbps) \\
		Java version & Open JDK 1.8.0\_222 \\
		MPI version & Intel MPI with MPJ-Express v0\_44 Java native bindings \\
		\bottomrule
	\end{tabular}
\end{table}